\documentclass{article}

\usepackage[german]{babel}
\usepackage[utf8]{inputenc}
\usepackage{amsmath}

\title{Ziffernblatt für CO$_2$-Anzeige}
\author{Ivo Blöchliger}

\begin{document}
\maketitle
Ein Messwert $m \in [0,1]$ soll in einen Winkel $w \in [0,1]$
umrechnet werden. Die Skala soll nicht-linear sein, so dass z.B. der
Wert $\overline{m}=\frac{1500-400}{5000-400} = \frac{11}{46}$ in der
Mitte platziert werden kann. (Skala von 400 bis 5000, 1500 in der
Mitte).


Wir suchen eine Funktion $f(x)$ so, dass $m = f(w)$ und $f^{-1}(m)=w$,
was schlussendlich implementiert werden soll.

Es gilt $f(0)=0$ und $f(1)=1$ und $f\left(\frac{1}{2}\right)=\overline{m}$.

Die Skalenverkürzung soll möglichst gleichmässig erfolgen, d.h.\
$f'(w)$ soll eine lineare Funktion sein. Also ist $f(w)$ eine
quadratische Funktion $f(w) = aw^2+bw+c$. Aus den obigen Bedingungen
folgt $c=0$ und
\[
a+b=1 \qquad \text{ und } \qquad \frac{1}{4}a+\frac{1}{2}b =
\overline{m},
\]
woraus mit Maxima
\begin{verbatim}
solve([a+b=1, a+2*b=4*m], [a,b]);
\end{verbatim}
\[
a = 2 - 4 m \qquad \text{ und } \qquad b = 4 m - 1
\]
folgt.

Daraus folgt mit Maxima
\begin{verbatim}
solve((2-4*m)*w^2+(4*m-1)*w=y, w);
\end{verbatim}
\[
f^{-1}(m) = {{\sqrt{\left(8-16\,\overline{m}\right)\,m+16\,\overline{m}^2-8\,\overline{m}+1}+
 4\,\overline{m}-1}\over{8\,\overline{m}-4}}.
\]
Für den Wert $\overline{m}=\frac{1500-400}{5000-400} = \frac{11}{46}$ erhält
man
\[
w = f^{-1}(m) = {{\sqrt{2208\,m+1}+1
  }\over{48}}
\]


\end{document}
